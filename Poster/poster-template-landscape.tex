% Template file for an a0 portrait poster.
% Written by Graeme, 2001-03 based on his SOC poster.
%
% See discussion and documentation at
% <http://www.astro.gla.ac.uk/users/norman/docs/posters/> 
%
%%%%%%%%%%%%%%%%%%%%%%%%%%%%%%%%%%%%%%%%
% Modified by Jozef Dobo\v{s} (c) 2011 % 
%%%%%%%%%%%%%%%%%%%%%%%%%%%%%%%%%%%%%%%%

\documentclass[a0,landscape]{a0poster}
% You might find the 'draft' option to a0 poster useful if you have
% lots of graphics, because they can take some time to process and
% display. (\documentclass[a0,draft]{a0poster})

\pagestyle{empty}
\setcounter{secnumdepth}{0}
\usepackage[absolute]{textpos}
\usepackage{txfonts}
\usepackage{wrapfig,times}
\usepackage{graphicx}
\usepackage{forloop}
\usepackage[margin=0cm]{geometry}

%%%%%%%%%%
% Colors %
%%%%%%%%%%
\usepackage{color}
\definecolor{TitleColor}{rgb}{1,1,1} % white
\definecolor{BannerOneColor}{rgb}{0,0,0} % pitch black
\definecolor{BannerTwoColor}{rgb}{0.93,0.08,0.31} % pinky red
\definecolor{BannerThreeColor}{rgb}{0,0.27,0.48} % dark blue
\definecolor{BannerFourColor}{rgb}{0.33,0.19,0.098} % brown
\definecolor{BannerSixColor}{rgb}{0,0.27,0.42} % dark blue
\definecolor{BannerSevenColor}{rgb}{0.62,0.77,0.86} % sky blue
\definecolor{BannerEightColor}{rgb}{0.35,0.33,0.01} % military green
\definecolor{BannerNineColor}{rgb}{0.85,0.86,0.34} % lime green
\definecolor{BannerTenColor}{rgb}{0,0.66,0.80} % strong blue
\definecolor{BannerElevenColor}{rgb}{0.46,0,0.20} % maroon
\definecolor{BannerTwelveColor}{rgb}{0.37,0.32,0.44} % dark washed violet
\definecolor{BannerThirteenColor}{rgb}{0.79,0.84,0.65} % light washed green
\definecolor{BannerFourteenColor}{rgb}{0.57,0.64,0.27} % dark washed green
\definecolor{BannerFifteenColor}{rgb}{0.92,0.91,0.88} % unusable washed 
\definecolor{BannerSixteenColor}{rgb}{0.94,0.36,0.14} % strong orange
\definecolor{BannerSeventeenColor}{rgb}{0.97,0.61,0.19} % orange
\definecolor{BannerEighteenColor}{rgb}{0.99,0.76,0.11} % mustard yellow
\definecolor{BannerNineteenColor}{rgb}{0.79,0.76,0.73} % light gray-ish
\definecolor{BannerTwentyColor}{rgb}{0.63,0.58,0.54} % dark gray-ish

%%%%%%%%%%%%%%%%%%%%%%%%%%%%%%%%%%%%%%%%%%%%%%%%%%%%%%
% Only change here to affect all headings            %
\newcommand{\headingcolor}{\color{BannerThreeColor}} 
%%%%%%%%%%%%%%%%%%%%%%%%%%%%%%%%%%%%%%%%%%%%%%%%%%%%%%

% see documentation for a0poster class for the size options here
\let\Textsize\normalsize
\def\Head#1{\noindent\hbox to \hsize{\hfil{\LARGE \headingcolor #1}}\bigskip}
\def\LHead#1{\noindent{\LARGE \headingcolor #1}\smallskip}
\def\Authors#1{\noindent{\LARGE #1}\smallskip}
\def\Subhead#1{\noindent{\large \headingcolor #1}}
\def\Title#1{\noindent{\VeryHuge #1}}


% Set up the grid
%
% Note that [0cm,0cm] is the margin round the edge of the page --
% it is _not_ the grid size. That is always defined as 
% PAGE_WIDTH/HGRID and PAGE_HEIGHT/VGRID. In this case we use
% 25 x 25. This gives us three wide columns for text (7 grid
% spacings) and four narrow columns (1 each) at each side of these 
% text columns
%
% Note however that texblocks can be positioned fractionally as well,
% so really any convenient grid size can be used.
%

% [margin, margin]{rows}{cols}
\TPGrid[0cm,0cm]{25}{25}  % 1 - 7 - 1 - 7 - 1 - 7 -1 Columns


% Mess with these as you like
\parindent=0pt
%\parindent=1cm
\parskip=0.5\baselineskip
\linespread{1.2}

% abbreviations
\newcommand{\ddd}{\,\mathrm{d}}

\begin{document}

% Understanding textblocks is the key to being able to do a poster in
% LaTeX. In
%
%    \begin{textblock}{width}(x,y)
%    ...
%    \end{textblock}
%
% the first argument gives the block width in units of the grid
% cells specified above in \TPGrid; the second gives the (x,y)
% position on the grid, with the y axis pointing down.

%%%%%%%%%%%%%%
% Top Banner %
%%%%%%%%%%%%%%
% if you change this part, you can get matching color for headings
% in Colors section above
\begin{textblock}{25}(0,0)
\includegraphics[width=\paperwidth]{banners/banner3.pdf}
\end{textblock}




%%%%%%%%%
% Title %
%%%%%%%%%
\begin{textblock}{23}(0.5,0.5)
{\color{TitleColor}
\Title{\LaTeXe\ Landscape Poster Template}\\\\
\Authors{Jozef Dobo\v{s}}\\
\texttt{\{j.dobos\}@cs.ucl.ac.uk}
}
\end{textblock}




% An example text block, to get you started!
\begin{textblock}{7}(1,3.5)
  \LHead{Introduction}
  
  Lorem ipsum dolor sit amet, consectetur adipiscing elit. Phasellus dignissim auctor semper. In vitae risus eu lacus varius blandit quis vel quam. Vestibulum hendrerit ligula lacus, non lacinia mi. Nulla congue nibh ut dui malesuada blandit. Nulla pulvinar nibh et enim volutpat fermentum. Cras ultrices viverra quam, sit amet elementum metus laoreet nec. Nam elementum nulla et metus malesuada tincidunt. Aliquam justo elit, malesuada lobortis fermentum vitae, ultricies sed neque. Maecenas molestie consequat nunc, at aliquet nunc scelerisque eu. Phasellus in ante massa. Etiam dapibus adipiscing sodales. Nulla eleifend sagittis arcu, eget viverra lacus pulvinar eget. 

 Suspendisse consectetur dignissim libero, at euismod justo mollis nec. Vestibulum lobortis varius est, ut facilisis enim tincidunt non. Vivamus eu metus massa, vel euismod eros. Sed sed ipsum sed metus pellentesque tincidunt. Suspendisse sollicitudin dictum est id sodales. Praesent adipiscing facilisis tellus vitae cursus. In a est velit, nec tempus lorem. Curabitur blandit tellus id libero malesuada sit amet dictum mauris tincidunt. Proin quis dui neque, nec posuere tellus. Pellentesque sed iaculis velit. Vivamus sit amet viverra magna. Curabitur interdum, massa et pretium sollicitudin, purus dui cursus purus, ut viverra lacus dolor eget dui. Mauris mattis, arcu at porttitor dapibus, lacus eros ornare orci, quis fringilla urna sapien aliquet justo. Morbi in lacus vitae dolor iaculis placerat. Maecenas vulputate consequat neque sed scelerisque. Sed ut justo risus. Quisque sem sem, condimentum ac vehicula nec, condimentum quis tortor. 

 Donec eget sollicitudin orci. Donec volutpat luctus ligula, non feugiat elit sagittis ac. Etiam id metus erat. Quisque leo nulla, varius quis accumsan at, malesuada nec mi. Ut ut quam at velit vehicula ornare et ut nibh. Donec et risus vel velit tincidunt tempus. Nulla ligula libero, consectetur a ultricies at, congue at enim. Suspendisse commodo vulputate diam, et luctus risus facilisis eu. Cras molestie eros vitae erat auctor eget tristique ipsum ornare. Morbi vitae mollis felis. Duis nibh enim, ullamcorper vitae luctus sed, blandit a dolor. Mauris eu ipsum lacus, id tincidunt dolor. Curabitur gravida, diam et iaculis tincidunt, arcu velit varius nibh, at gravida leo augue ut elit. Duis mollis interdum tincidunt. Mauris eu turpis sed neque rutrum aliquam. 

  \[
  x' = \frac{1}{x}
  \]

\end{textblock}



\begin{textblock}{7}(9,3.5)
\newcounter{int}
\begin{enumerate}
\forloop{int}{1}{\value{int} < 21}{
\item \includegraphics[width=0.98\linewidth]{banners/banner\arabic{int}.pdf}
}
\end{enumerate}
\end{textblock}


% Another text block in the bottom right.
\begin{textblock}{7}(17,18)
  \LHead{Conclusion}
  
  Lorem ipsum dolor sit amet, consectetur adipiscing elit. Phasellus dignissim auctor semper. In vitae risus eu lacus varius blandit quis vel quam. Vestibulum hendrerit ligula lacus, non lacinia mi. Nulla congue nibh ut dui malesuada blandit. Nulla pulvinar nibh et enim volutpat fermentum. Cras ultrices viverra quam, sit amet elementum metus laoreet nec. Nam elementum nulla et metus malesuada tincidunt. Aliquam justo elit, malesuada lobortis fermentum vitae, ultricies sed neque. Maecenas molestie consequat nunc, at aliquet nunc scelerisque eu. Phasellus in ante massa. Etiam dapibus adipiscing sodales. Nulla eleifend sagittis arcu, eget viverra lacus pulvinar eget. 

\end{textblock}



% If you want to add a figure do something like this:

%\begin{textblock}{3}(1,15)
%  \begin{center}
%  \resizebox{3\TPHorizModule}{!}{\includegraphics{images/group-logo.pdf}}
%\\{\bfseries Figure 5:} caption
%  \end{center}
%\end{textblock}

\begin{textblock}{9}(3.5,21.2)
\LHead{Bottom Line}

It is possible to do a landscape poster!

\end{textblock}



% Place the group logo at the bottom left - visually this balances
% well with the University logo at the top right. 
\begin{textblock}{4}(0.5,23)    
%\resizebox{1.5\TPHorizModule}{!}{
\includegraphics{images/group-logo.pdf}
%}
\end{textblock}

\end{document}

